\documentclass{proc}

\begin{document}

\title{Alignment or lack thereof in Charts and it's Impact}

\author{You, Me}

\maketitle

\section{Introduction}

Within the chart, alignment is used to equate datapoints which have equivalent values on explicit dimensions.
But when we think of visualization in general, things we might not think of as datapoints still should be aligned.

\begin{itemize}
\item to facilitate consumption
\item to ease the cognitive burden
\item to allow people to interpret them as related
\end{itemize}

\section{One-sentence description}

\section{Project Type}

\section{Audience} 
\begin{quote}
\textit{Who is the audience for this project? 
How does it meet their needs? 
What happens if their needs remain unmet?}
\end{quote}

\section{Approach}
\subsection{Details}
\begin{quote}
\textit{What is your approach?}
\end{quote}

\subsection{Evidence for Success}
\begin{quote}
\textit{Why do you think it will work?} 
\end{quote}


\section{Best-case Impact Statement}
\begin{quote}
\textit{In the best-case scenario, what would be the impact statement (conclusion statement) for this project? \cite{wijk2005value, pike2009science}}
\end{quote}

\section{Major Milestones}

\section{Obstacles}

\subsection{Major obstacles} % (if these fail, the project is over)

\subsection{Minor obstacles}

\section{Resources Needed}
\begin{quote}
\textit{What additional resources do you need to complete this project?}
\end{quote}

\section{5 Related Publications}
\begin{quote}
\textit{List 5 major publications that are most relevant to this project, and how they are related (sample citation \cite{wijk2005value}).}
\end{quote}

\section{Define Success}
\begin{quote}
\textit{What is the minimum amount of work necessary for this work be publishable?}
\end{quote}

\bibliographystyle{abbrv}
\bibliography{../bib/mqp}
\end{document}
